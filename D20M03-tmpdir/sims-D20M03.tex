% Options for packages loaded elsewhere
\PassOptionsToPackage{unicode}{hyperref}
\PassOptionsToPackage{hyphens}{url}
%
\documentclass[
]{article}
\usepackage{amsmath,amssymb}
\usepackage{iftex}
\ifPDFTeX
  \usepackage[T1]{fontenc}
  \usepackage[utf8]{inputenc}
  \usepackage{textcomp} % provide euro and other symbols
\else % if luatex or xetex
  \usepackage{unicode-math} % this also loads fontspec
  \defaultfontfeatures{Scale=MatchLowercase}
  \defaultfontfeatures[\rmfamily]{Ligatures=TeX,Scale=1}
\fi
\usepackage{lmodern}
\ifPDFTeX\else
  % xetex/luatex font selection
\fi
% Use upquote if available, for straight quotes in verbatim environments
\IfFileExists{upquote.sty}{\usepackage{upquote}}{}
\IfFileExists{microtype.sty}{% use microtype if available
  \usepackage[]{microtype}
  \UseMicrotypeSet[protrusion]{basicmath} % disable protrusion for tt fonts
}{}
\makeatletter
\@ifundefined{KOMAClassName}{% if non-KOMA class
  \IfFileExists{parskip.sty}{%
    \usepackage{parskip}
  }{% else
    \setlength{\parindent}{0pt}
    \setlength{\parskip}{6pt plus 2pt minus 1pt}}
}{% if KOMA class
  \KOMAoptions{parskip=half}}
\makeatother
\usepackage{xcolor}
\usepackage[margin=1in]{geometry}
\usepackage{color}
\usepackage{fancyvrb}
\newcommand{\VerbBar}{|}
\newcommand{\VERB}{\Verb[commandchars=\\\{\}]}
\DefineVerbatimEnvironment{Highlighting}{Verbatim}{commandchars=\\\{\}}
% Add ',fontsize=\small' for more characters per line
\usepackage{framed}
\definecolor{shadecolor}{RGB}{248,248,248}
\newenvironment{Shaded}{\begin{snugshade}}{\end{snugshade}}
\newcommand{\AlertTok}[1]{\textcolor[rgb]{0.94,0.16,0.16}{#1}}
\newcommand{\AnnotationTok}[1]{\textcolor[rgb]{0.56,0.35,0.01}{\textbf{\textit{#1}}}}
\newcommand{\AttributeTok}[1]{\textcolor[rgb]{0.13,0.29,0.53}{#1}}
\newcommand{\BaseNTok}[1]{\textcolor[rgb]{0.00,0.00,0.81}{#1}}
\newcommand{\BuiltInTok}[1]{#1}
\newcommand{\CharTok}[1]{\textcolor[rgb]{0.31,0.60,0.02}{#1}}
\newcommand{\CommentTok}[1]{\textcolor[rgb]{0.56,0.35,0.01}{\textit{#1}}}
\newcommand{\CommentVarTok}[1]{\textcolor[rgb]{0.56,0.35,0.01}{\textbf{\textit{#1}}}}
\newcommand{\ConstantTok}[1]{\textcolor[rgb]{0.56,0.35,0.01}{#1}}
\newcommand{\ControlFlowTok}[1]{\textcolor[rgb]{0.13,0.29,0.53}{\textbf{#1}}}
\newcommand{\DataTypeTok}[1]{\textcolor[rgb]{0.13,0.29,0.53}{#1}}
\newcommand{\DecValTok}[1]{\textcolor[rgb]{0.00,0.00,0.81}{#1}}
\newcommand{\DocumentationTok}[1]{\textcolor[rgb]{0.56,0.35,0.01}{\textbf{\textit{#1}}}}
\newcommand{\ErrorTok}[1]{\textcolor[rgb]{0.64,0.00,0.00}{\textbf{#1}}}
\newcommand{\ExtensionTok}[1]{#1}
\newcommand{\FloatTok}[1]{\textcolor[rgb]{0.00,0.00,0.81}{#1}}
\newcommand{\FunctionTok}[1]{\textcolor[rgb]{0.13,0.29,0.53}{\textbf{#1}}}
\newcommand{\ImportTok}[1]{#1}
\newcommand{\InformationTok}[1]{\textcolor[rgb]{0.56,0.35,0.01}{\textbf{\textit{#1}}}}
\newcommand{\KeywordTok}[1]{\textcolor[rgb]{0.13,0.29,0.53}{\textbf{#1}}}
\newcommand{\NormalTok}[1]{#1}
\newcommand{\OperatorTok}[1]{\textcolor[rgb]{0.81,0.36,0.00}{\textbf{#1}}}
\newcommand{\OtherTok}[1]{\textcolor[rgb]{0.56,0.35,0.01}{#1}}
\newcommand{\PreprocessorTok}[1]{\textcolor[rgb]{0.56,0.35,0.01}{\textit{#1}}}
\newcommand{\RegionMarkerTok}[1]{#1}
\newcommand{\SpecialCharTok}[1]{\textcolor[rgb]{0.81,0.36,0.00}{\textbf{#1}}}
\newcommand{\SpecialStringTok}[1]{\textcolor[rgb]{0.31,0.60,0.02}{#1}}
\newcommand{\StringTok}[1]{\textcolor[rgb]{0.31,0.60,0.02}{#1}}
\newcommand{\VariableTok}[1]{\textcolor[rgb]{0.00,0.00,0.00}{#1}}
\newcommand{\VerbatimStringTok}[1]{\textcolor[rgb]{0.31,0.60,0.02}{#1}}
\newcommand{\WarningTok}[1]{\textcolor[rgb]{0.56,0.35,0.01}{\textbf{\textit{#1}}}}
\usepackage{graphicx}
\makeatletter
\def\maxwidth{\ifdim\Gin@nat@width>\linewidth\linewidth\else\Gin@nat@width\fi}
\def\maxheight{\ifdim\Gin@nat@height>\textheight\textheight\else\Gin@nat@height\fi}
\makeatother
% Scale images if necessary, so that they will not overflow the page
% margins by default, and it is still possible to overwrite the defaults
% using explicit options in \includegraphics[width, height, ...]{}
\setkeys{Gin}{width=\maxwidth,height=\maxheight,keepaspectratio}
% Set default figure placement to htbp
\makeatletter
\def\fps@figure{htbp}
\makeatother
\setlength{\emergencystretch}{3em} % prevent overfull lines
\providecommand{\tightlist}{%
  \setlength{\itemsep}{0pt}\setlength{\parskip}{0pt}}
\setcounter{secnumdepth}{-\maxdimen} % remove section numbering
\ifLuaTeX
  \usepackage{selnolig}  % disable illegal ligatures
\fi
\usepackage{bookmark}
\IfFileExists{xurl.sty}{\usepackage{xurl}}{} % add URL line breaks if available
\urlstyle{same}
\hypersetup{
  pdftitle={Experiment05 code raw},
  hidelinks,
  pdfcreator={LaTeX via pandoc}}

\title{Experiment05 code raw}
\author{}
\date{\vspace{-2.5em}2025-03-20}

\begin{document}
\maketitle

First we generate the parameters for simulation:

\begin{Shaded}
\begin{Highlighting}[]
\CommentTok{\# Load libraries}
\FunctionTok{library}\NormalTok{(tictoc)}
\FunctionTok{library}\NormalTok{(tidyverse)}
\FunctionTok{library}\NormalTok{(tidyr)}

\NormalTok{tictoc}\SpecialCharTok{::}\FunctionTok{tic}\NormalTok{(}\StringTok{"Section 0: Total running time"}\NormalTok{)}
\NormalTok{tictoc}\SpecialCharTok{::}\FunctionTok{tic}\NormalTok{(}\StringTok{"Section 1: Time for Parameter Generation:"}\NormalTok{)}

\CommentTok{\# ==== Generate parameters ====}
\CommentTok{\# Generate experiment name}
\NormalTok{generate\_id }\OtherTok{\textless{}{-}} \ControlFlowTok{function}\NormalTok{() \{}
\NormalTok{  day }\OtherTok{\textless{}{-}} \FunctionTok{format}\NormalTok{(}\FunctionTok{Sys.Date}\NormalTok{(), }\StringTok{"\%d"}\NormalTok{)}
\NormalTok{  month }\OtherTok{\textless{}{-}} \FunctionTok{format}\NormalTok{(}\FunctionTok{Sys.Date}\NormalTok{(), }\StringTok{"\%m"}\NormalTok{)}
  
  \CommentTok{\# char\_string \textless{}{-} paste0(sample(c(LETTERS, 0:9), 4, replace = TRUE), collapse = "")}
  \FunctionTok{paste0}\NormalTok{(}\StringTok{"Exp05"}\NormalTok{, }\StringTok{"{-}D"}\NormalTok{, day, }\StringTok{"M"}\NormalTok{, month)}
\NormalTok{\}}

\NormalTok{wd }\OtherTok{\textless{}{-}} \StringTok{"/mnt/atgc{-}d3/sur/users/mrivera/glv{-}research"}
\NormalTok{exp\_id }\OtherTok{\textless{}{-}} \FunctionTok{generate\_id}\NormalTok{()}
\NormalTok{params\_path }\OtherTok{\textless{}{-}} \FunctionTok{file.path}\NormalTok{(wd, }\StringTok{"Data"}\NormalTok{, }\FunctionTok{paste0}\NormalTok{(exp\_id, }\StringTok{".tsv"}\NormalTok{))}
\end{Highlighting}
\end{Shaded}

Generate grid of parameters:

\begin{Shaded}
\begin{Highlighting}[]
\CommentTok{\# params\_to\_sim \textless{}{-} expand\_grid(n\_species = rep(c(20, 100), times = 300), p\_neg = 1, p\_noint = seq(0, 1, by = 0.05)) \%\textgreater{}\%}
\CommentTok{\#   mutate(id = ids::random\_id(n = length(n\_species), bytes = 5)) \%\textgreater{}\%}
\CommentTok{\#   mutate(x0\_seed = as.vector(sample(1:1e6, length(n\_species), replace = FALSE))) \%\textgreater{}\%}
\CommentTok{\#   mutate(mu\_seed = as.vector(sample(1:1e6, length(n\_species),replace = FALSE))) \%\textgreater{}\%}
\CommentTok{\#   mutate(A\_seed = as.vector(sample(1:1e6, length(n\_species), replace = FALSE)))}
\CommentTok{\# }
\CommentTok{\# \# Verify if ids are unique and in case they are, save the parameters.}
\CommentTok{\# if (nrow(params\_to\_sim) == length(unique(params\_to\_sim$id))) \{}
\CommentTok{\#   \# Save Parameters as TSV}
\CommentTok{\#   data.table::fwrite(x = params\_to\_sim, file = params\_path, sep = "\textbackslash{}t")}
\CommentTok{\#   message("\textbackslash{}nParameteres generated and saved with ID: ",  paste0(exp\_id, ".tsv"), "\textbackslash{}n")}
\CommentTok{\# \} else \{}
\CommentTok{\#   message("\textbackslash{}nAn error ocurred. IDs are not unique\textbackslash{}n")}
\CommentTok{\# \}}

\CommentTok{\# Read TSV file}
\NormalTok{params\_to\_sim }\OtherTok{\textless{}{-}}\NormalTok{ data.table}\SpecialCharTok{::}\FunctionTok{fread}\NormalTok{(params\_path, }\AttributeTok{sep =} \StringTok{"}\SpecialCharTok{\textbackslash{}t}\StringTok{"}\NormalTok{)}
\NormalTok{tictoc}\SpecialCharTok{::}\FunctionTok{toc}\NormalTok{()}
\end{Highlighting}
\end{Shaded}

We split the data into chunks of \texttt{n\_cores}:

\begin{Shaded}
\begin{Highlighting}[]
\CommentTok{\# ==== Split data into chunks====}
\NormalTok{tictoc}\SpecialCharTok{::}\FunctionTok{tic}\NormalTok{(}\StringTok{"Section 2: Divide data into chunks"}\NormalTok{)}
\FunctionTok{library}\NormalTok{(parallel)}
\NormalTok{num\_cores }\OtherTok{\textless{}{-}}\NormalTok{ parallel}\SpecialCharTok{::}\FunctionTok{detectCores}\NormalTok{() }\SpecialCharTok{{-}} \DecValTok{1}  \CommentTok{\# Use one less than the total number of cores}
\FunctionTok{cat}\NormalTok{(}\StringTok{"The number of cores that will be used are: "}\NormalTok{, num\_cores, }\StringTok{"}\SpecialCharTok{\textbackslash{}n}\StringTok{"}\NormalTok{)}

\NormalTok{split\_table }\OtherTok{\textless{}{-}} \ControlFlowTok{function}\NormalTok{(df, n\_chunks) \{}
  \FunctionTok{split}\NormalTok{(df, }\FunctionTok{cut}\NormalTok{(}\FunctionTok{seq\_len}\NormalTok{(}\FunctionTok{nrow}\NormalTok{(df)), }\AttributeTok{breaks =}\NormalTok{ n\_chunks, }\AttributeTok{labels =} \ConstantTok{FALSE}\NormalTok{))}
\NormalTok{\}}

\NormalTok{chunks }\OtherTok{\textless{}{-}} \FunctionTok{split\_table}\NormalTok{(params\_to\_sim, num\_cores)}
\FunctionTok{message}\NormalTok{(}\StringTok{"}\SpecialCharTok{\textbackslash{}n}\StringTok{Data split completed...}\SpecialCharTok{\textbackslash{}n}\StringTok{"}\NormalTok{)}
\NormalTok{tictoc}\SpecialCharTok{::}\FunctionTok{toc}\NormalTok{() }\CommentTok{\# For section 2}
\end{Highlighting}
\end{Shaded}

We create a separate directory for each core to prevent race conditions
(when two cores access the same directory simultaneously).

\begin{Shaded}
\begin{Highlighting}[]
\CommentTok{\# ==== Generate directories for each core ====}
\NormalTok{tictoc}\SpecialCharTok{::}\FunctionTok{tic}\NormalTok{(}\StringTok{"Section 3: Generate directories for each core"}\NormalTok{)}

\CommentTok{\# Generate workers directories}
\NormalTok{ode\_dir }\OtherTok{\textless{}{-}} \FunctionTok{file.path}\NormalTok{(wd, }\StringTok{"Results"}\NormalTok{, exp\_id, }\StringTok{"Parallel{-}02"}\NormalTok{)}

\CommentTok{\# Function to create main and worker directories}
\NormalTok{create\_dirs }\OtherTok{\textless{}{-}} \ControlFlowTok{function}\NormalTok{(main\_dir, num\_cores) \{}
  \ControlFlowTok{if}\NormalTok{ (}\SpecialCharTok{!}\FunctionTok{dir.exists}\NormalTok{(main\_dir)) }\FunctionTok{dir.create}\NormalTok{(main\_dir, }\AttributeTok{recursive =} \ConstantTok{TRUE}\NormalTok{)}
  
  \CommentTok{\# Create worker directories}
\NormalTok{  worker\_dirs }\OtherTok{\textless{}{-}} \FunctionTok{file.path}\NormalTok{(main\_dir, }\FunctionTok{paste0}\NormalTok{(}\StringTok{"worker\_"}\NormalTok{, }\FunctionTok{seq\_len}\NormalTok{(num\_cores)))}
  \FunctionTok{invisible}\NormalTok{(}\FunctionTok{lapply}\NormalTok{(worker\_dirs, dir.create, }\AttributeTok{showWarnings =} \ConstantTok{FALSE}\NormalTok{))}
  
  \FunctionTok{return}\NormalTok{(worker\_dirs)}
\NormalTok{\}}
\CommentTok{\# Create directories }
\NormalTok{workers\_ODE }\OtherTok{\textless{}{-}} \FunctionTok{create\_dirs}\NormalTok{(ode\_dir, num\_cores)}
\FunctionTok{message}\NormalTok{(}\StringTok{"}\SpecialCharTok{\textbackslash{}n}\StringTok{Working directories created at path:}\SpecialCharTok{\textbackslash{}n}\StringTok{"}\NormalTok{, ode\_dir,}\StringTok{"}\SpecialCharTok{\textbackslash{}n}\StringTok{"}\NormalTok{)}
\NormalTok{tictoc}\SpecialCharTok{::}\FunctionTok{toc}\NormalTok{() }\CommentTok{\# For section 3}
\end{Highlighting}
\end{Shaded}

We source the function to generate the gLV parameters using the initial
parameters and for solve it:

\begin{Shaded}
\begin{Highlighting}[]
\CommentTok{\# Source function to regenerate parameters}
\FunctionTok{source}\NormalTok{(}\StringTok{"/mnt/atgc{-}d3/sur/users/mrivera/glv{-}research/GIT{-}gLV/D20M03{-}tmpdir/gLV{-}Params.R"}\NormalTok{)}
\CommentTok{\# print(regenerate)}
\end{Highlighting}
\end{Shaded}

\begin{Shaded}
\begin{Highlighting}[]
\CommentTok{\# Source function to solve gLV equations}
\FunctionTok{source}\NormalTok{(}\StringTok{"/mnt/atgc{-}d3/sur/users/mrivera/glv{-}research/GIT{-}gLV/D20M03{-}tmpdir/solve{-}gLV.R"}\NormalTok{)}
\CommentTok{\# print(solve\_gLV)}
\end{Highlighting}
\end{Shaded}

\begin{Shaded}
\begin{Highlighting}[]
\CommentTok{\# Source function to generate extinctions}
\FunctionTok{source}\NormalTok{(}\StringTok{"/mnt/atgc{-}d3/sur/users/mrivera/glv{-}research/GIT{-}gLV/D20M03{-}tmpdir/extinctions{-}fun.R"}\NormalTok{)}
\CommentTok{\# print(simulate\_all\_extinctions)}
\end{Highlighting}
\end{Shaded}

We wrap the code for parallelizing the simulations.

\begin{Shaded}
\begin{Highlighting}[]
\CommentTok{\# ==== Wrapper for running all required steps ====}
\NormalTok{parallel.sims }\OtherTok{\textless{}{-}} \ControlFlowTok{function}\NormalTok{(index, path\_core) \{}
  
  \CommentTok{\# Generate parameters}
\NormalTok{  params }\OtherTok{\textless{}{-}} \FunctionTok{regenerate}\NormalTok{(index)}
  
  \CommentTok{\# Run simulation}
\NormalTok{  output\_ode }\OtherTok{\textless{}{-}} \FunctionTok{solve\_gLV}\NormalTok{(}\AttributeTok{times =} \DecValTok{1000}\NormalTok{, params)}
  
  \CommentTok{\# Run extinctions}
\NormalTok{  ext\_tb }\OtherTok{\textless{}{-}} \FunctionTok{simulate\_all\_extinctions}\NormalTok{(output\_ode, params)}
  
  \CommentTok{\# Define paths}
\NormalTok{  path\_ode }\OtherTok{\textless{}{-}} \FunctionTok{file.path}\NormalTok{(path\_core, }\FunctionTok{paste0}\NormalTok{(}\StringTok{"O\_"}\NormalTok{, params}\SpecialCharTok{$}\NormalTok{id, }\StringTok{".tsv"}\NormalTok{))}
\NormalTok{  path\_ext }\OtherTok{\textless{}{-}} \FunctionTok{file.path}\NormalTok{(path\_core, }\FunctionTok{paste0}\NormalTok{(}\StringTok{"Ext\_"}\NormalTok{, params}\SpecialCharTok{$}\NormalTok{id, }\StringTok{".rds"}\NormalTok{))}
  
  \CommentTok{\# Calculate NAs}
\NormalTok{  NA\_count }\OtherTok{\textless{}{-}} \FunctionTok{sum}\NormalTok{(}\FunctionTok{is.na}\NormalTok{(output\_ode))}
  
  \CommentTok{\# Save simulation}
\NormalTok{  utils}\SpecialCharTok{::}\FunctionTok{write.table}\NormalTok{(output\_ode, }\AttributeTok{file =}\NormalTok{ path\_ode, }\AttributeTok{sep =} \StringTok{"}\SpecialCharTok{\textbackslash{}t}\StringTok{"}\NormalTok{, }\AttributeTok{row.names =} \ConstantTok{FALSE}\NormalTok{, }\AttributeTok{col.names =} \ConstantTok{TRUE}\NormalTok{)}
  \FunctionTok{saveRDS}\NormalTok{(ext\_tb, }\AttributeTok{file =}\NormalTok{ path\_ext)}
  
  \FunctionTok{return}\NormalTok{(}\FunctionTok{c}\NormalTok{(}\AttributeTok{id =}\NormalTok{ params}\SpecialCharTok{$}\NormalTok{id, }\AttributeTok{NA\_count =}\NormalTok{ NA\_count))}
\NormalTok{\}}
\end{Highlighting}
\end{Shaded}

We Parallelize the code and get the \texttt{NA\ counting}.

\begin{Shaded}
\begin{Highlighting}[]
\CommentTok{\# ==== Parallelize it ====}
\NormalTok{tictoc}\SpecialCharTok{::}\FunctionTok{tic}\NormalTok{(}\StringTok{"Section 4: Run simulations and extinctions using the parallel package"}\NormalTok{)}

\NormalTok{nas\_counts }\OtherTok{\textless{}{-}}\NormalTok{ parallel}\SpecialCharTok{::}\FunctionTok{mclapply}\NormalTok{(}\DecValTok{1}\SpecialCharTok{:}\NormalTok{num\_cores, }\ControlFlowTok{function}\NormalTok{(core\_id) \{}
  
  \FunctionTok{message}\NormalTok{(}\StringTok{"Starting worker "}\NormalTok{, core\_id, }\StringTok{"....}\SpecialCharTok{\textbackslash{}n}\StringTok{"}\NormalTok{)}
  
\NormalTok{  core\_chunk }\OtherTok{\textless{}{-}}\NormalTok{ chunks[[core\_id]]  }\CommentTok{\# rows assigned to this core}
  
\NormalTok{  na.vec }\OtherTok{\textless{}{-}} \FunctionTok{lapply}\NormalTok{(}\DecValTok{1}\SpecialCharTok{:}\FunctionTok{nrow}\NormalTok{(core\_chunk), }\ControlFlowTok{function}\NormalTok{(i) \{}
    \FunctionTok{parallel.sims}\NormalTok{(core\_chunk[i, ], }\AttributeTok{path\_core =}\NormalTok{ workers\_ODE[core\_id])}
\NormalTok{  \})}
  
  \FunctionTok{message}\NormalTok{(}\StringTok{"Ending worker "}\NormalTok{, core\_id, }\StringTok{"....}\SpecialCharTok{\textbackslash{}n}\StringTok{"}\NormalTok{)}
  
  \FunctionTok{return}\NormalTok{(na.vec)}
  
\NormalTok{\}, }\AttributeTok{mc.cores =}\NormalTok{ num\_cores)}
\NormalTok{tictoc}\SpecialCharTok{::}\FunctionTok{toc}\NormalTok{() }\CommentTok{\# For section 4}


\CommentTok{\# ==== Get NAs number on simulations====}
\NormalTok{tictoc}\SpecialCharTok{::}\FunctionTok{tic}\NormalTok{(}\StringTok{"Section 5: Count total number of NAs"}\NormalTok{)}
\NormalTok{counts\_df }\OtherTok{\textless{}{-}} \FunctionTok{as.data.frame}\NormalTok{(}\FunctionTok{matrix}\NormalTok{(}\FunctionTok{unlist}\NormalTok{(nas\_counts), }\AttributeTok{ncol =} \DecValTok{2}\NormalTok{, }\AttributeTok{byrow =} \ConstantTok{TRUE}\NormalTok{))}
\FunctionTok{colnames}\NormalTok{(counts\_df) }\OtherTok{\textless{}{-}} \FunctionTok{unique}\NormalTok{(}\FunctionTok{names}\NormalTok{(nas\_counts))}

\CommentTok{\# Save Parameters as TSV}
\NormalTok{na\_path }\OtherTok{\textless{}{-}} \FunctionTok{file.path}\NormalTok{(wd, }\StringTok{"Results"}\NormalTok{, exp\_id, }\StringTok{"Na{-}count{-}02.tsv"}\NormalTok{)}
\NormalTok{data.table}\SpecialCharTok{::}\FunctionTok{fwrite}\NormalTok{(}\AttributeTok{x =}\NormalTok{ counts\_df, }\AttributeTok{file =}\NormalTok{ na\_path, }\AttributeTok{sep =} \StringTok{"}\SpecialCharTok{\textbackslash{}t}\StringTok{"}\NormalTok{)}
\NormalTok{tictoc}\SpecialCharTok{::}\FunctionTok{toc}\NormalTok{() }\CommentTok{\# For section 5}
\end{Highlighting}
\end{Shaded}

We create symbolic links of the simulation\ldots{}

\begin{Shaded}
\begin{Highlighting}[]
\CommentTok{\# ==== Create symbolic links====}
\NormalTok{tictoc}\SpecialCharTok{::}\FunctionTok{tic}\NormalTok{(}\StringTok{"Section 6: Generate symbolic links"}\NormalTok{)}

\FunctionTok{source}\NormalTok{(}\StringTok{"/mnt/atgc{-}d3/sur/users/mrivera/glv{-}research/GIT{-}gLV/D20M03{-}tmpdir/Forge\_symlinks.R"}\NormalTok{)}

\CommentTok{\# Define source and target directories}
\NormalTok{workers\_path }\OtherTok{\textless{}{-}} \FunctionTok{file.path}\NormalTok{(wd, }\StringTok{"Results"}\NormalTok{, exp\_id, }\StringTok{"Parallel{-}02"}\NormalTok{)}
\NormalTok{outs\_path }\OtherTok{\textless{}{-}} \FunctionTok{file.path}\NormalTok{(wd, }\StringTok{"Results"}\NormalTok{, exp\_id, }\StringTok{"Outputs{-}02"}\NormalTok{)}
\NormalTok{exts\_path }\OtherTok{\textless{}{-}} \FunctionTok{file.path}\NormalTok{(wd, }\StringTok{"Results"}\NormalTok{, exp\_id, }\StringTok{"Extinctions{-}02"}\NormalTok{)}
\FunctionTok{generate\_symlinks}\NormalTok{(workers\_path, outs\_path, exts\_path)}
\NormalTok{tictoc}\SpecialCharTok{::}\FunctionTok{toc}\NormalTok{() }\CommentTok{\# For section 6}
\NormalTok{tictoc}\SpecialCharTok{::}\FunctionTok{toc}\NormalTok{() }\CommentTok{\# For Total running time}
\end{Highlighting}
\end{Shaded}


\end{document}
